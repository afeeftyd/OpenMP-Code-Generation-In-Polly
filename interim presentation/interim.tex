\documentclass{beamer}

\usetheme{default}
\usepackage{fancyvrb,relsize}

\usepackage{amsmath}
\usepackage{bbding}
\usepackage{courier}
\usepackage{enumerate}
\setbeamertemplate{itemize items}[circle]
\usepackage{graphicx}
% \usepackage{graphics}
%\usepackage{epstopdf}

\mode<presentation>
{
\setbeamertemplate{footline}
{\rightline{\insertframenumber/\inserttotalframenumber}}
}

\title{A Framework For Realizing Autoparallelism
Through Automatic OpenMP Code Generation
 }
\author{Raghesh A (CS09M032)}
%\date

\begin{document}

% slide
\begin{frame}
\titlepage
\end{frame}

% slide
\begin{frame}{Outline}
\begin{itemize}
\item Background
\item Introduction to LLVM and Polly
\item Polly - Paper Accepted at IMPACT
\item Automatic Code Generation
\item Workdone
\item Testing
\item Results Obtained
\item Work Plan
\end{itemize}
\end{frame}

% slide
\begin{frame}{Background}
\begin{itemize}
\item Auto parallelism
\item The Polyhedral model
\item Manual OpenMP Code Generation
\end{itemize}
\end{frame}

\begin{frame}{Introduction to LLVM and Polly}
\begin{itemize}
\item LLVM (Low Level Virtual Machine)
	\begin{itemize}
	\item Framework for implementing compilers
	\item Common low level code repersentation
	\item Lifelong analysis and transformation of programs
	\end{itemize}
\item Polly (Polyhedral Optimization in LLVM)
	\begin{itemize}
	\item Implementing Polyhedral Optimization in LLVM
	\item Effort towards Auto Parallelism in programs.
	\end{itemize}
\end{itemize}
\end{frame}

% slide
\begin{frame}{Automatic OpenMP Code Generation}
\begin{itemize}
\item Auto parallelism
\end{itemize}
\end{frame}

\end{document}



