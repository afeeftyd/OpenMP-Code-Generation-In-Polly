\documentclass[a4paper,12pt]{article}
%\usepackage{fancyvrb,relsize}
\usepackage[small,compact]{titlesec}
\usepackage[pdftex]{graphicx}
\usepackage{listings}
\lstset{language=C}
%\usepackage[margin=3.50cm]{geometry}
\linespread{1.5}
\setlength{\parindent}{10pt}
\setlength{\parskip}{0.85ex plus 0.65ex minus 0.3ex}
\sloppy
\setlength{\oddsidemargin}{0in} \setlength{\evensidemargin}{0in}
\setlength{\textwidth}{6.5in} \setlength{\textheight}{9.5in}
\setlength{\topmargin}{-0.65in}

\begin{document}

\begin{center}
{\bf {\LARGE Chapter 1}\linebreak\linebreak{Background}}
\linebreak
\linebreak
\end{center}

\section{Auto Parallelization}

We can take the advantage of hardware support for parallelism only if the compiler has support for
generating the parallel code. There are interfaces like OpenMP for developing parallel applications.
But the user has to manually provide the annotations for it in the source code. This becomes
a tedious task for the user and he has to ensure the correctness of the code too.
This prompted researchers to explore mechanisms for finding out the parallel portions
of the code without manual intervention.

\section{The Polyhedral Model}

Polyhedral optimizations are used for many kind of memory access optimization by
looking into the memory access pattern of any piece of code. Any kind of classical
 loop optimization techniques like tiling can be used for this purpose. 



\section{LLVM}
LLVM defines a common, low-level code representation in Static Single Assignment
(SSA) form, with several novel features. The LLVM compiler framework and code
representation together provide a combination of key capabilities that are
important for practical, lifelong analysis and transformation of programs.
One of the important features of LLVM is that the output of all the
transformation passes have same intermediate representation(LLVM IR), which
makes the programmer’s life simple.

\section{Polly}
The framework discussed in this document is implemented using Polly[polly],
an open source[licence] compiler  optimization framework that uses a mathematical
 representation, the polyhedral model, to represent and transform loops and other
 control flow structures. It is an effort towards achieving autoparallelism in programs.
 The transformations are being implemented in LLVM(Low level virtual machine). 
Polly can detect parallel loops, issue vector instructions and generate OpenMP code(focus of 
this document) corresponding to those loops. Polly try to expose more parallelism
with the help of polyhedral model. A loop which does not look parallel can be transformed
to a parallel loop and these can be vectorized or parallelize using OpenMP.

More details on LLVM and Polly can be found at chapters 2 and 3 respectively.

\section{Manual OpenMP Code Generation}

\section{SPEC2006 Benchmarks}
\end{document}
