\label{chap:openmp}

Optimizations in Polly should be able to create loops that are executed in parallel, as if the user would have
added some OpenMP pragmas. To achieve this, code generation needs to emit code that calls an OpenMP
library to be executed in parallel. The GNU OpenMP Library(libgomp) is used for this purpose. The
dependency analysis module of Polly automatically detects parallel loops(SCoPs) and are given to OpenMP
code generation module. Here we generate the required libgomp library calls. The code generated is similar
the code generated if the user have added OpenMP pragmas[2]. The following sections explain the steps
taken towards generating the OpenMP code. The generated code is in LLVM IR format.
\section{Generating OpenMP Library Calls}
Consider the for loop below to have a basic understanding about what is to be done.

{\footnotesize
\begin{lstlisting}
  for (int i = 0; i <= N; i++)
    A[i] = 1 ;
\end{lstlisting}
}

The above for loop is detected as a parallel loop and given for OpenMP code generation. Here the following
sequence of GOMP library calls with proper arguments and return types(signature) has to be generated in
LLVM IR format.

\begin{itemize}
\item GOMP parallel loop runtime start
\item subfunction
\item GOMP parallel end
\end{itemize}

The code for body of the for loop is generated inside the subfunction which has the following GOMP library
calls to achieve the necessary parallelism.

\begin{itemize}
\item GOMP loop runtime next
\item GOMP loop end nowait
\end{itemize}

The signature and descriptions of each of the above functions can be found in in libgomp manual[3].

\section{Support for inner loops}
So far OpenMP code created apply only for outer loops, which is detected as SCoP. Next step is to do it for
inner loops. Due to dependency issues the outer loop is not detected as SCoP, but innerloop can be safely
parallelized as in the following example.

{\footnotesize
\begin{lstlisting}
  for (int i = 0; i < N; i++)
    for (int j = 0; j < N; j++)
      A[j] += i;
\end{lstlisting}
}

Those loops need the values of the surrounding induction variables in the OpenMP subfunction. We need
to pass the values of the outer induction variables in a structure to the subfunction. This step is almost
completed with some minor issues to be fixed.

