\subsection{Tree Transducer with Insertion System}

I have implemented three different types of tree transducers \emph{(Top-down, Extended Top-down and 
Extended Multi Bottom-up)} as a part of my M.Tech project and have shown some language translation like
 \emph{english to malayalam, malayalam to english and arabic to english}.

But my implementation works only for fixed trees. We can give a sentence in the form of a tree, say \emph{initial tree} and some 
transitions as inputs to the transducer. Based on the transitions given, the transducer converts given sentence to some other
language.\\For example a sentence \emph{The machine runs the program} can be given in the form of a tree as
\begin{figure}[h]
\begin{center}
{\tiny
\pstree[nodesep=1pt,levelsep=3ex]{\Tr{S}}
{
	\pstree[nodesep=1pt,levelsep=3ex]{\Tr{NP}}
	{
		\pstree[nodesep=1pt,levelsep=3ex]{\Tr{DET}}
		{
			\Tr{The}
		}
		\pstree[nodesep=1pt,levelsep=3ex]{\Tr{N}}
		{
			\Tr{machine}
		}
	}
        \pstree[nodesep=1pt,levelsep=3ex]{\Tr{VP}}
	{
        	\pstree[nodesep=1pt,levelsep=3ex]{\Tr{V}}
		{
			\Tr{runs}
		}					
	        \pstree[nodesep=1pt,levelsep=3ex]{\Tr{NP}}
		{
	        	\pstree[nodesep=1pt,levelsep=3ex]{\Tr{DET}}
			{
				\Tr{the}
			}
		        \pstree[nodesep=1pt,levelsep=3ex]{\Tr{N}}
			{
				\Tr{program}
			}
		}
	}
}
\label{fig2}
\caption{Initial Tree}
%\noindent \rule{\textwidth}{1pt}
}
\end{center}
\end{figure}

What happens if we need some changes in the input sentence, after generating the tree?\\\\
One idea is to create a new tree for the new sentence. But it is not an efficient method, if there are only minor changes.
Because a small change may not affect the whole tree which already generated. So reconstruction of the tree is not a good idea.\\
Here I would like to use the idea of \emph{tree insdel system}. The idea is to construct a tree, say 
\emph{auxilary trees} for the parts of sentence where changes occurs. This auxilaruy trees can be inserted to the initial trees
 using the concept of tree insdel system.\\
For example the tree structure of the sentence \emph{The machine which compiles the language runs the program} is almost similar 
to the above tree. By making some insertions in the above tree we can generate tree for this sentence. To do that, first 
construct an auxilary tree for \emph{which compiles the language} and insert it to the initial tree.
\begin{figure}[h]
\begin{center}
{\tiny
%\begin{minipage}[b]{0.2\linewidth}
\pstree[nodesep=1pt,levelsep=2ex]{\Tr{S}}
{
	\pstree[nodesep=1pt,levelsep=3ex]{\Tr{NP}}
	{
		\pstree[nodesep=1pt,levelsep=3ex]{\Tr{RE}}
		{
			\Tr{which}
		}
	}
        \pstree[nodesep=1pt,levelsep=2ex]{\Tr{VP}}
	{
        	\pstree[nodesep=1pt,levelsep=2ex]{\Tr{V}}
		{
			\Tr{compiles}
		}					
	        \pstree[nodesep=1pt,levelsep=2ex]{\Tr{NP}}
		{
	        	\pstree[nodesep=1pt,levelsep=2ex]{\Tr{DET}}
			{
				\Tr{the}
			}
		        \pstree[nodesep=1pt,levelsep=2ex]{\Tr{N}}
			{
				\Tr{language}
			}
		}
	}
}
\label{fig3}
\caption{Auxilary Tree}
%\noindent \rule{\textwidth}{1pt}
}
\end{center}
\end{figure}

\begin{figure}[h]
\begin{center}
{\tiny
\pstree[nodesep=1pt,levelsep=3ex]{\Tr{S}}
{
	\pstree[nodesep=1pt,levelsep=3ex]{\Tr{NP}}
	{
		\pstree[nodesep=1pt,levelsep=3ex]{\Tr{NP}}
		{
			\pstree[nodesep=1pt,levelsep=3ex]{\Tr{DET}}
			{
				\Tr{The}
			}
			\pstree[nodesep=1pt,levelsep=3ex]{\Tr{N}}
			{
				\Tr{machine}
			}
		}
   		\pstree[nodesep=1pt,levelsep=3ex]{\Tr{S}}
		{
            \psset{linestyle=dashed}
            {      
    			\pstree[nodesep=1pt,levelsep=3ex]{\Tr{NP}}
	    		{
		    		\pstree[nodesep=1pt,levelsep=3ex]{\Tr{RE}}
			    	{
				    	\Tr{which}
    				}
	    		}
		        \pstree[nodesep=1pt,levelsep=3ex]{\Tr{VP}}
			    {
		        	\pstree[nodesep=1pt,levelsep=3ex]{\Tr{V}}
				    {
					    \Tr{compiles}
    				}					
			        \pstree[nodesep=1pt,levelsep=3ex]{\Tr{NP}}
	    			{
			        	\pstree[nodesep=1pt,levelsep=3ex]{\Tr{DET}}
		    			{
			    			\Tr{the}
				    	}
				        \pstree[nodesep=1pt,levelsep=3ex]{\Tr{N}}
				    	{
					    	\Tr{language}
					    }
       				}
        		}
		     }
        }
	}
    \pstree[nodesep=1pt,levelsep=3ex]{\Tr{VP}}
	{
        	\pstree[nodesep=1pt,levelsep=3ex]{\Tr{V}}
		{
			\Tr{runs}
		}					
	        \pstree[nodesep=1pt,levelsep=3ex]{\Tr{NP}}
		{
	        	\pstree[nodesep=1pt,levelsep=3ex]{\Tr{DET}}
			{
				\Tr{the}
			}
		        \pstree[nodesep=1pt,levelsep=3ex]{\Tr{N}}
			{
				\Tr{program}
			}
		}
	}
}
\label{fig4}
\caption{Initial Tree after Insertion of Auxilary Tree}
%\noindent \rule{\textwidth}{1pt}
}
\end{center}
\end{figure}
The system mentined above has a lot of applications in language processing. 
So I would like to do some work on that.\\
