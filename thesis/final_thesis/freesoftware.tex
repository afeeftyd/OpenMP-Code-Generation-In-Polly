Doing projects in free software world is always fun. More than fun this is a platform where
we experience the benefits of sharing knowledge. Here I would like to share some points 
regarding this in light of my experience with Polly.

One seeking for free software project should own an open mind. It is very difficult to
contribute for one who have fear in sharing their ideas and knowledge. There is no need for
such a fear since we will always get the ownership of our contribution. Only thing is that
others who are willing to contribute can freely update our code. There is no harm in
this approach. We are opening the doors for others and make the functionality
of our code better and in turn improving the quality of our product.

This is a perfect platform for academic and student community to work. For a student to
make contributions to a project there should have good opportunities to learn the
current status of the work. This information is hidden in case of proprietary and closed software.
The proprietary approach will just provide some interfaces here and there to add a
new functionality. In some cases the student will not ever know about the project as a whole and
will never know where it is deployed.

What makes us qualified for contributing to free software projects? In fact the answer is
that we require only some basic qualification. Rest will be acquired during the development period.
The very first one would be some experience in basic build tools like an editor(vi, emacs, etc.),
make, gdb, etc, proficiency in one or two programming languages and of course willingness to learn.
Almost all people in the community are ready to offer their helping hands to one who has this
basic qualification. Another prerequisite is to have good communication skills. That is expressing
yourself through media like mailing lists. Ensure that you follow basic mailing list ethics(like avoiding
usage of short words ('d' for 'the')). 

The stepping stone is to convince the community that we have the basic qualifications discussed
above and make confidence about you. Here is a tip for this. Set up the environment yourself
and start to learn the code. You can get help through the
mailing list at any time. If possible tackle a bug and try to fix that. Having done this you will
be in a position to add new features to the project. While we are implementing something and if we
are stuck somewhere always give a try to solve the problem ourselves and then seek help from
the developers with all your observations about the issue. This will boost up their confidence
in you and they will definitely help us. The reality is that the community needs more and more
people who are willing to contribute. They will never discourage you. Once you became an active
developer it is always good to review others code and post our comments on it. Make sure that
our code is reviewed by others also.

To conclude, the four freedoms defined by free software foundation\footnote{\url{http://www.gnu.org/philosophy/free-sw.html}}
enables you to strengthen your knowledge and skills. There is nobody to stop you. Enjoy the freedom !!
