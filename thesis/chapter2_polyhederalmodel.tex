\documentclass[a4paper,12pt]{book}
%\usepackage{fancyvrb,relsize}
\usepackage[small,compact]{titlesec}
\usepackage[pdftex]{graphicx}
\usepackage{listings}
\lstset{language=C}
%\usepackage[margin=3.50cm]{geometry}
%\linespread{1.5}
%\setlength{\parindent}{0pt}
\setlength{\parskip}{0.85ex plus 0.65ex minus 0.3ex}
\sloppy
%\setlength{\oddsidemargin}{0in} \setlength{\evensidemargin}{0in}
%\setlength{\textwidth}{6.5in} \setlength{\textheight}{9.5in}
%\setlength{\topmargin}{-0.65in}

\begin{document}
There are different types optimizations that can be performed on a program to improve its
performance. The optimization can be made for finding data locality and hence extracting
parallelism. Starting from the early history of programming languages the internal representation
of program is done with Abstract Syntax Tree(AST). Though some elementary transformation can
be performed on AST it is tough to carry out complex transformations like dependencies among
statements inside a loop. Trees are very rigid data structures to do such transformations.
In this chapter a extremely powerful mathematical model which puts together analysis power,
expressiveness and flexibility is explained in detail.
%\begin{center}
\chapter{The Polyhedral Model}
%{\bf {\LARGE Chapter 1}\linebreak\linebreak{Background}}
%\linebreak
%\linebreak
%\end{center}

\end{document}
