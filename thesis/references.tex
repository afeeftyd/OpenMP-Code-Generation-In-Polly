Chapter 1
---------

https://computing.llnl.gov/tutorials/pthreads/
https://computing.llnl.gov/tutorials/parallel_comp/
http://software.intel.com/en-us/articles/threading-models-for-high-performance-computing-pthreads-or-openmp/
http://www.google.com/url?sa=t&source=web&cd=4&ved=0CDEQFjAD&url=http%3A%2F%2Fwww.cs.uiowa.edu%2F~ebavier%2Fmain.article.pdf&ei=WPxwTfedPInNrQei-oTSCg&usg=AFQjCNFrgEzpHr63OQaqQ9JrNE9UNZCzvA

Poly Hedral Model
----------------

1. Clan - A Polyhedral Representation Extractor for High Level Programs Edition 1.0, for Clan 1.0.0
2. A conservative approach to handle full functions in Polyhedral model. Cedric Bastoul
3. Codegeneration in the Polyhedral Model is easier than you think.
@InProceedings{Bas04b,
    author =   {C\'{e}dric Bastoul},
    title =      {Code Generation in the Polyhedral Model Is Easier Than You Think},
    booktitle =    {PACT'13 IEEE International Conference on Parallel Architecture
        and Compilation Techniques},
    year =   2004,
    pages =    {7--16},
    month =  {September},
    address =  {Juan-les-Pins, France}
}

4. GRAPHITE: Polyhedral Analyses and Optimizations for GCC
LLVM
----

3. http://llvm.org/pubs/2004-01-30-CGO-LLVM.html
@InProceedings{LLVM:CGO04,
    author    = {Chris Lattner and Vikram Adve},
    title     = "{LLVM: A Compilation Framework for Lifelong Program Analysis \& Transformation}",
    booktitle = "{Proceedings of the 2004 International Symposium on Code Generation and Optimization (CGO'04)}",
    address   = {Palo Alto, California},
    month     = {Mar},
    year      = {2004}
}

4. http://llvm.org/pubs/2002-12-LattnerMSThesis.html

@MastersThesis{Lattner:MSThesis02,
    author  = {Chris Lattner},
    title   = "{LLVM: An Infrastructure for Multi-Stage Optimization}",
    school  = "{Computer Science Dept., University of Illinois at Urbana-Champaign}",
    year    = {2002},
    address = {Urbana, IL},
    month   = {Dec},
    note    = {{\em See {\tt http://llvm.cs.uiuc.edu}.}}
}

5. http://llvm.org/docs/tutorial/

6. http://llvm.org/docs/Passes.html

7. http://llvm.org/docs/TestingGuide.html

8. http://llvm.org/docs/TestingGuide.html

9. http://llvm.org/docs/ProgrammersManual.html

10. http://llvm.org/docs/LangRef.html

11. http://llvm.org/docs/WritingAnLLVMPass.html

Auto Parallelization
--------------------
12. Towards a holistic approach to auto-parallelization: integrating profile-driven parallelism detection and machine-learning based mapping

@inproceedings{Tournavitis:2009:THA:1542476.1542496,
    author = {Tournavitis, Georgios and Wang, Zheng and Franke, Bj\"{o}rn and O'Boyle, Michael F.P.},
    title = {Towards a holistic approach to auto-parallelization: integrating profile-driven parallelism detection and machine-learning based mapping},
    booktitle = {Proceedings of the 2009 ACM SIGPLAN conference on Programming language design and implementation},
    series = {PLDI '09},
    year = {2009},
    isbn = {978-1-60558-392-1},
    location = {Dublin, Ireland},
    pages = {177--187},
    numpages = {11},
    url = {http://doi.acm.org/10.1145/1542476.1542496},
    doi = {http://doi.acm.org/10.1145/1542476.1542496},
    acmid = {1542496},
    publisher = {ACM},
    address = {New York, NY, USA},
    keywords = {auto-parallelization, machine-learning based parallelism mapping, openmp, profile-driven parallelism detection},
} 


13.Automatic parallelization for multi-core processors
Part of doctoral dissertation, The Ohio State University, Sep 
2006 -- Aug 2008
Funded in part by US National Science Foundation (NSF) under 
grants 0121676 and 0509467

The polyhedral model for compiler optimization is a framework that 
provides a powerful abstraction to reason about program transformations 
by viewing a dynamic instance (iteration) of each of the program's 
statements as an integer point in a well-defined space which is the 
statement's  polyhedron, with dimensions of the polyhedron corresponding 
to loops surrounding the statement. With such a representation for each 
statement and a precise characterization of inter or intra-statement 
dependences, it is possible to reason about the correctness of a sequence 
of complex loop transformations using machinery from linear programming 
and linear algebra.

Affine transformations in the polyhedral model capture a complex 
sequence of execution-reordering loop transformations that can improve 
performance by parallelization as well as locality enhancement. Although 
a significant body of research has addressed affine scheduling and 
partitioning, the problem of automatically finding good affine transforms 
for communication-optimized coarse-grained parallelization together with 
locality optimization for the general case of arbitrarily-nested loop 
sequences had remained a challenging problem.

We developed a new automatic transformation framework to
optimize sequences of imperfectly-nested loops for parallelism and 
locality simultaneously. The approach works by finding good ways of 
tiling through a powerful, practical, and scalable cost function embedded 
in an Integer Linear Programming formulation. The approach enables 
minimization of inter-tile communication volume in the processor space, 
             and minimization of reuse distances for local execution at each node.  
             Programs requiring one-dimensional versus multi-dimensional time 
             schedules (with scheduling-based approaches) are all handled with the 
             same algorithm.  Synchronization-free parallelism, permutable loops 
             or pipelined parallelism, and inner parallelism can be detected[8].


14. U. Bondhugula, A. Hartono, J. Ramanujam, and P. Sadayappan, "A Practical and Automatic Polyhedral Program Optimization System," Proc. ACM SIGPLAN 2008 Conference on Programming Language Design and Implementation (PLDI 08), Tucson, June 2008

15. U. Bondhugula, M. Baskaran, S. Krishnamoorthy, J. Ramanujam, A. Rountev, and P. Sadayappan, "Automatic Transformations for Communication-Minimized Parallelization and Locality Optimization in the Polyhedral Model," in Proc. CC 2008 - International Conference on Compiler Construction, Budapest, Hungary, March-April 2008

16.         author = {Uday Bondhugula and Albert Hartono
    and J. Ramanujam and P. Sadayappan},
    title = {A Practical Automatic Polyhedral Program Optimization System},
    booktitle = {ACM SIGPLAN Conference on Programming Language Design and 
        Implementation (PLDI)},
        year = 2008,
        month = jun
        }

17. Graphite Papers

18. Tobias's website docs


Profiling
---------

SPEC
----
http://www.spec.org/cpu2006/Docs/

POLLYBENCH
----------
Automatic Parallelization in a Binary Rewriter
Aparna Kotha, Kapil Anand, Matthew Smithson, Greeshma Yellareddy and Rajeev Barua
Department of Electrical and Computer Engineering
University of Maryland, College Park, 2074

